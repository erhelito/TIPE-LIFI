\documentclass[french,titlepage]{extarticle}
\usepackage[T1]{fontenc}
\usepackage[latin9]{luainputenc}
\usepackage{geometry}
\geometry{verbose,tmargin=3cm,bmargin=3cm,lmargin=3cm,rmargin=3cm}
\usepackage{babel}
\PassOptionsToPackage{normalem}{ulem}
\usepackage{ulem}
\usepackage[]
 {hyperref}

\makeatletter
%%%%%%%%%%%%%%%%%%%%%%%%%%%%%% User specified LaTeX commands.
\usepackage{hyperref}   % pour les liens cliquables de la table des matières

\usepackage[svgnames,dvipsnames]{xcolor}

\makeatletter
\@addtoreset{section}{part}   % reprendre à partir de 1 les sections des parties suivantes.
\makeatother

\makeatother

\begin{document}
\begin{center}
{\LARGE\textbf{\uline{Étude de la communication par laser à grande
distance}}}{\LARGE\par}
\par\end{center}

Dans un monde où les informations circulent de plus en plus vite,
il est essentiel de développer des méthodes innovantes pour transmettre
des données de manière efficace, sécurisée et fiable. C'est pourquoi
j'ai choisi de réaliser un travail de recherche sur la communication
par la lumière à grande distance à l'air libre (sans fibre optique),
en utilisant un laser comme moyen de transmission. L'objectif de ce
projet est de concevoir et de mettre en œuvre un système capable de
transmettre des images ou un signal sonore entre deux bâtiments séparés
d'une centaine de mètres, en utilisant la lumière comme support de
transmission.

\section*{Ancrage au thème}

Conversion d'un signal électrique en signal lumineux et vice-versa,
transmission dans l'atmosphère.

\section*{Positionnement thématique}

\emph{PHYSIQUE (électronique), PHYSIQUE (optique)}

\section*{Mots-clés}
\begin{itemize}
\item \emph{Laser}
\item \emph{Filtrage}
\item \emph{CAN}
\item \emph{Communication sans fil}
\end{itemize}

\section*{Bibliographie commentée}

Les technologies de communication sans fil ont connu une évolution
rapide ces dernières années, propulsées par l\textquoteright augmentation
du nombre de dispositifs connectés et la nécessité de répondre à une
demande croissante de données à haut débit. Face à la saturation des
spectres radiofréquences, la technologie Light Fidelity (Li-Fi), qui
repose sur l\textquoteright utilisation de la lumière visible pour
la transmission de données, s\textquoteright impose comme une alternative
prometteuse au Wi-Fi traditionnel. Contrairement aux réseaux sans
fil basés sur les ondes radio, le Li-Fi exploite le spectre lumineux,
offrant ainsi une capacité de transmission largement supérieure \cite{key-7}.

Le Li-Fi repose sur des sources lumineuses telles que les diodes électroluminescentes
(LED) ou des lasers pour la transmission de données, et utilise la
modulation de lumière pour encoder l\textquoteright information. Cette
technologie permet des vitesses de transmission pouvant atteindre
jusqu\textquoteright à 1 Gb/s, et tire profit de la vaste bande passante
disponible dans la lumière visible, estimée à 2600 fois plus grande
que celle des ondes radio \cite{key-17}. De plus, contrairement aux
technologies traditionnelles, le Li-Fi n\textquoteright interfère
pas avec les dispositifs utilisant les radiofréquences, ce qui le
rend particulièrement adapté à des environnements saturés ou à des
situations nécessitant une faible interférence \cite{key-7}.

En outre, cette technologie assure un niveau de sécurité accru, car
la lumière visible ne traverse pas les murs et ne peut être captée
qu\textquoteright à proximité de la source lumineuse, rendant ainsi
les transmissions moins vulnérables aux interceptions à distance \cite{key-21}.

Cependant, bien que le Li-Fi soit particulièrement adapté aux transmissions
de données à courte portée (généralement de quelques mètres à quelques
dizaines de mètres), il présente certaines limitations liées à la
nécessité d\textquoteright une ligne de vue entre l\textquoteright émetteur
et le récepteur. Cette contrainte limite son utilisation dans des
environnements extérieurs ou dans des situations où des obstacles
peuvent bloquer la transmission. À cet égard, la communication par
laser, qui repose sur des principes similaires au Li-Fi, permet d\textquoteright élargir
la portée des transmissions jusqu\textquoteright à des distances de
100 mètres ou plus, tout en maintenant un haut débit de données \cite{key-17}.

Dans les environnements à forte densité de trafic, comme les zones
urbaines ou les autoroutes, la capacité de communication par laser
à couvrir de longues distances sans interférer avec d\textquoteright autres
systèmes sans fil est un atout majeur. En outre, le laser offre des
possibilités uniques dans des applications telles que les communications
entre véhicules, où la précision et la directionnalité de la transmission
sont cruciales pour garantir la sécurité et la fiabilité des systèmes
de transport intelligent \cite{key-21}.

En conclusion, le Li-Fi et la communication par laser représentent
des solutions de communication sans fil complémentaires, répondant
aux défis de capacité, de sécurité et de gestion des interférences
dans des environnements de plus en plus saturés. Grâce à l\textquoteright utilisation
de l\textquoteright infrastructure lumineuse existante et aux avancées
technologiques, ces technologies ouvrent la voie à de nouvelles applications
dans des secteurs clés tels que la géolocalisation, la sécurité des
transports, et la transmission de données à haut débit dans des espaces
urbains. Leur adoption généralisée nécessitera néanmoins un effort
concerté de la part des chercheurs et des industriels pour surmonter
les obstacles techniques et normatifs encore présents.

Dans l'objectif d'étudier la communication par laser et ses limites,
nous réaliserons un dispositif capable d'émettre une information modulée
à une vitesse de transmission satisfaisante, ainsi qu'un système en
mesure de recevoir cette information à une distance de l'ordre de
la centaine de mètres, tout en assurant la bonne qualité de l'information
sur le lieu de réception.

\section*{Problématique retenue}

Peut-on utiliser la technologie laser pour communiquer sur de longues
distances sans utilisation de fibre optique ?

\section*{Objectifs du TIPE}
\begin{enumerate}
\item Concevoir un système d'émission de signal électronique via un signal
optique
\item Concevoir un système de réception du signal optique via signal électronique
\item Réaliser un dispositif capable de communiquer à l'échelle de la dizaine
de mètres
\item Concevoir un système d'encodage de l'information efficace limitant
la perte d'information
\item Amélioration du système pour une utilisation à longue distance
\end{enumerate}
\begin{thebibliography}{1}
\bibitem{key-7}Louiza Hamada. Conception d\textquoteright une architecture
suintroductions :     Louiza hamada :         Ces derniers temps,
le nombre de gadgets et d'applications reposant sur la communication
sans fil a explosé, entraînant des transformations considérables dans
le style de vie des gens. Pour gérer cette augmentation de connectivité,
plusieurs techniques avancées sont prévues avec les normpportant la
technologie Li-Fi. Réseaux et télécommunications {[}cs.NI{]}. Université
de Haute Alsace - Mulhouse, 2022. Français. NNT : 2022MULH5026. tel-03955897

\bibitem{key-8}\href{https://www.photoniques.com/articles/photon/pdf/2017/03/photon201786p22.pdf}{https://www.photoniques.com/articles/photon/pdf/2017/03/photon201786p22.pdf}

\bibitem{key-17}Technologie LiFi (Light Fidelity) par Luc CHASSAGNE,
professeur, Laboratoire LISV EA4048, université de Versailles Saint-Quentin,
Vélizy, France

\bibitem{key-21}Alin Cailean. Etude et réalisation d\textquoteright un
système de communications par lumière visible (VLC/LiFi). Application
au domaine automobile. Optique / photonique. Université de Versailles
Saint-Quentin en Yvelines, 2014. Français. NNT : tel-01156468

\end{thebibliography}

\end{document}
